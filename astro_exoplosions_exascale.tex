\documentclass[11pt,twocolumn]{article}

\usepackage{mathpazo}

\usepackage[margin=1in]{geometry}

\usepackage[small,compact]{titlesec}                                            
%\titlespacing{\subsection}{0pt}{*2}{*0.125}       

\begin{document}

\begin{center}
{\sffamily \bfseries \large Modeling Astrophysical Explosions with Sustained Exascale Computing} \\
Michael Zingale\footnotemark[1], Alan Calder\footnotemark[1]
\end{center}

\footnotetext[1]{Stony Brook University}

Our understanding of stars and their fates is based on coupling
observations to theoretical models.  Unlike laboratory physicists, we
cannot perform experiments on stars (e.g.\ tweaking the energy
generation or setting two stars on a collision course), but rather must
patiently take what nature allows us to observe.  Simulation offers a means
of virtual experimentation, enabling a detailed understanding of the
most violent ongoing explosions in the Universe---the death of stars.

Stars can explode in a surprising variety of ways, driven by either
nuclear or gravitational potential energy release.  The
explosion can consume the entire star (or stars) or just
surface layers.  Exotic remnants, like neutron stars or black holes, or
simply nothing can be left behind.  Stellar explosions are critically
important sources of nucleosynthesis---for example, all of the iron in
the Universe results from these explosions, enriching the interstellar
medium in heavy elements. 

Astrophysics has been at the forefront of high-performance computing
for decades, with simulations of supernovae explosions being one of
the prime foci.  Both DOE (including the national labs) and
NSF have supported the development and application of simulation codes
to stellar explosions.  Many successes have been met, but there are
still great uncertainties in the mechanisms of core collapse
supernovae (the death of massive stars) and thermonuclear supernovae
(the explosion of compact white dwarf stars).  Astronomy is also
entering the ``big data'' era, with survey telescopes like LSST coming
online toward the end of the decade that will greatly expand
observations of astrophysical transients, perhaps finding entire new
classes.

Cutting edge research in stellar astrophysics is being done with both
large multidimensional simulations, demanding large core counts, as
well as suites of one-dimensional evolutionary simulations with
exceptionally detailed microphysics.  The interplay between these two
paths is critical to building up a physical picture of stellar
explosions, and each has unique (and increasing) computational
demands.  For brevity, we focus here on examples from
multi-dimensional work.

The standard practice in stellar astrophysics is to describe the star
as a fluid and use domain decomposition to divide the work across
computational nodes.  Ideally, shared memory parallelism is used
within a node, reducing memory overheads.  Work on exploiting
accelerators (GPUs and Phi processors) is underway for many codes, and
standard technologies (MPI, OpenMP, and OpenACC) promise to allow for
portability.

An example of a ``success" in computational astrophysics is the
enormous progress made over the last decade in understanding 
thermonuclear (type Ia) supernovae. Large-scale simulation 
allowed exploration of a variety of progenitor systems and
explosion mechanisms, allowing researcher to address not
just the question of a successful explosion, but deeper issues such
as systematic effects on the brightness of an event and explanations
for unusual or outlying events.  These simulations require the
interaction of many different researchers and simulation codes
capable of modeling the different phases.

\paragraph*{Increasing computing power.}

An increase of 100$\times$ in computing power will allow capability
simulations at unprecedented fidelity. Fluid flows are chaotic, and a
range of instabilities and turbulence are ever present in our models
of exploding stars. It was only recently that simulations switched
from being predominantly two-dimensional to
three-dimensional---enabled by the large increasing in computing power
over the last decade.  The goal of extant simulations is to understand
the feasibility of different theoretical models for explosions and to
probe the physics of the explosion mechanism itself.  However, the
range of length and timescales in stars that can be captured through
simulation is a small fraction of the true scales in stars.  This
means that approximations are made either explicitly (through the
introduction of subgrid scale models) or implicitly (having a
numerical dissipation that operates on much large scales than nature
would have).  The common way to test these approximations is to do
convergence studies, where the resolution of a simulation is changed
(over at most a decade) seeking convergence of the qualitative
results.  A looming question though is whether this is some
resolution, yet unobtainable, where the nature of the solution will
change qualitatively.  The promised increase in computing power will
allow for a much greater range in length scales to be modeled.

The increase in computer time also will allow for an expansion of the
physics modeled.  Today, small nuclear reaction networks ($\sim$
10--20 nuclei) are used, which approximately capture the energetics of
the flow, but are unable to make detailed predictions of the
nucleosynthetic yields.  Additional pieces, like radiation transport
are crudely approximated (if modeled at all).  The increase in
computational power will allow for more realistic physics throughout.

Finally, such an increase in computing time would bring today's
capability computing into the realm of routine, allowing investigators
to perform many times the present volume of production simulations.
In astrophysics, this increase will allow suites of simulations 
investigate the sensitivity of events on both model and physical
parameters, enabling formal uncertainty quantification at 
unprecedented levels.  This is true of both the multi-dimensional
and one-dimensional simulations.

\paragraph*{Impacts.} In addition to answering questions about the 
astrophysics, research into modeling astrophysical explosions at the
exascale will significantly impact many fields of science that address
multi-scale, multi-physics applications.  The simulation codes
developed for these problems have application to terrestial combustion
phenomena, climate and atmospherical flows, and applications related
to the DOE laboratory interests.  Importantly, this research provides
an excellent training group for the next generation of computational
scientistis, who can take their training to other disiplinces.


\paragraph*{Capabilities needed.}  Computational fluid dynamics requires 
high performance interconnects, as nearest neighbor (and global for
some physics) communication is needed each timestep.  This means that
traditional supercomputers are required over simple clusters.  A major
challenge with the increase in simulation size is the analysis of the
data (100s of TB per simulation).  In situ data analysis will need to
take on a bigger role in the future---many of the pieces for this are
coming into place.

Many times, the targeted use of supercomputing resources is using a
significant fraction of the machine for ``hero'' calculations.
However, science often needs capacity computing---many parameter
studies of the system help us understand the robustness of our models.
Going forward, there is a need for both capacity and leadership-class
computing centers.


\paragraph*{Foundational issues.}

Developing, maintaining, and supporting simulation codes takes
considerable effort, and often this work does is not given the same
recognition and rewards as the scientific results themselves.  This
puts the code developers, especially those early in their careers, at
a competitive disadvantage. Increased support for code development and
community support through the traditional grant process would greatly
help to capitalize on code investments.  Open source codes also greatly
help amortize the costs of code development, and enable (and
encourage) reproducibility of results, a hallmark of science.  The
astrophysics community does a reasonable job in making codes available, and
incentive structures should be setup to further encourage this.

A further difficulty is that awards of computer time don't come with
monetary support for the researchers who will run and analyze the
simulations, and grants don't come with a guarantee of computer time,
so there is a chicken-and-egg problem, and a necessity of having to
have both in place independently to make the best use of either
resource.

Finally, continued support for summer schools to train students is
essential (the Argonne Training Program on Extreme-Scale Computing is
an excellent example).



\end{document}
